\documentclass[$if(fontsize)$$fontsize$,$endif$$if(lang)$$lang$,$endif$$if(papersize)$$papersize$,$endif$$for(classoption)$$classoption$$sep$,$endfor$]{$documentclass$}

% envioronment detect
% Xelatex is the engine?or pdflatex or luatex
% https://www.ctan.org/tex-archive/macros/generic/ifxetex
\usepackage{ifxetex,ifluatex}

% deal with compatible problem that is early package
% http://texdoc.net/texmf-dist/doc/latex/base/fixltx2e.pdf
\usepackage{fixltx2e} % provides \textsubscript


% layout
\usepackage[a4paper, top=2.5cm, bottom=2.5cm, left=3cm, right=2.5cm]{geometry}



$if(highlighting-macros)$
    $highlighting-macros$
$endif$

$if(verbatim-in-note)$
    \usepackage{fancyvrb}
    \VerbatimFootnotes
$endif$


\usepackage{indentfirst}
%\usepackage[toc,page]{appendix}
%\usepackage[dutch]{babel}
\usepackage[toc,page]{appendix}


\usepackage{cleveref}

\makeatletter
\newcommand\appendix@section[1]{%
  \refstepcounter{section}%
  % arabic 123, Alph ABC, Roman I II
  \orig@section*{附錄 \@arabic\c@section: #1\centering}%
  \addcontentsline{toc}{section}{附錄 \@arabic\c@section: #1}%
}
\let\orig@section\section
\g@addto@macro\appendix{\let\section\appendix@section}
\makeatother
% for math equation display
\usepackage{amssymb,amsmath}
\ifxetex
    % mathspec 是以 fontspec 為後端,可以在 XeLaTeX 文稿中選擇使用特定的其他數學字型。
    \usepackage{mathspec}
    \usepackage{xltxtra,xunicode}
\else
    \usepackage{fontspec}
\fi
\defaultfontfeatures{Mapping=tex-text,Scale=MatchLowercase}

\usepackage{mathptmx}

% there the english and chinese font will be parameter
\usepackage[indentfirst=false]{xeCJK}
\defaultfontfeatures{Mapping=tex-text}
% 設定中英文的字型
% 字型的設定可以使用系統內的字型,而不用像以前一樣另外安裝
% set chinese font
\setCJKmainfont{ukai.ttc}
% set english font
\setmainfont{Times New Roman}

% 中文自動換行
\XeTeXlinebreaklocale "zh"
% 文字的彈性間距
\XeTeXlinebreakskip = 0pt plus 1pt
\newfontlanguage{Chinese}{CHN}

% can add non-formatted text , or render colorful word
$if(listings)$
    \usepackage{listings}
    \lstset{
       breaklines=true,
    }
$endif$

% pandoc will treat the document as literate Haskell source
% http://pandoc.org/demo/example19/Literate-Haskell-support.html
$if(lhs)$
    \lstnewenvironment{code}{\lstset{language=Haskell,basicstyle=\small\ttfamily}}{}
    
    
$endif$

%修改標題名稱
\renewcommand{\figurename}{圖}
\renewcommand{\tablename}{表}
\renewcommand\contentsname{目錄}
\renewcommand\listfigurename{圖目錄}
\renewcommand\listtablename{表目錄}
\renewcommand{\abstractname}{Abstract}
\renewcommand\bibname{參考文獻}
\newcommand{\loflabel}{圖}
\newcommand{\lotlabel}{表}


% font setup
\usepackage{sectsty}
\usepackage{setspace}

% 章次20級,節次16級,小節次以下14級,本文12級字
\def\LARGE{\fontsize{20}{30}\selectfont}%章次
\def\Large{\fontsize{16}{24}\selectfont}%節次
\def\large{\fontsize{14}{21}\selectfont}%小節次
\chapterfont{\LARGE}
\sectionfont{\Large}
\subsectionfont{\large}

\usepackage{titlesec}
\usepackage{titletoc}

% A pack­age pro­vid­ing an in­ter­face to sec­tion­ing com­mands for se­lec­tion from var­i­ous ti­tle styles.
% E.g., marginal ti­tles and to change the font of all head­ings with a sin­gle com­mand, also pro­vid­ing
% sim­ple one-step page styles. Also in­cludes a pack­age to change the page styles when there are floats in a page.
% You may as­sign head­ers/foot­ers to in­di­vid­ual floats, too.
\titlespacing{\chapter}{0pt}{*0}{*2}
\titlespacing{\section}{0pt}{*1}{*1}
\titlespacing{\subsection}{0pt}{*1}{*1}
\titlespacing{\subsubsection}{0pt}{*1}{*1}


% reset the chapter format
\usepackage{CJKnumb}
\titleformat{\chapter}[hang]{\bfseries\LARGE\centering}{第\CJKnumber{\arabic{chapter}}章}{0.5em}{}
\titlecontents{chapter}[0em] {}{\normalfont\normalsize\bfseries\makebox[4.1em][l] {第\CJKnumber{\thecontentslabel}章}}{} {\titlerule*[0.7pc]{.}\contentspage}

% if linestretch variable is set up
% enable below script

% line space
$if(linestretch)$
    \usepackage{setspace}
    \setstretch{$linestretch$}
$endif$



% 設定行距
\linespread{1}\selectfont


% paragraph setting

\setlength{\parindent}{1cm}
\setlength{\parskip}{6pt plus 2pt minus 1pt}
\setlength{\emergencystretch}{3em}  % prevent overfull lines
\newcommand{\tightlist}{%
\setlength{\itemsep}{0pt}\setlength{\parskip}{0pt}}

\usepackage{indentfirst}

% Paragraph set


% Redefines (sub)paragraphs to behave more like sections
\let\oldparagraph\paragraph
\renewcommand{\paragraph}[1]{\oldparagraph{#1}\mbox{}}
\let\oldsubparagraph\subparagraph
\renewcommand{\subparagraph}[1]{\oldsubparagraph{#1}\mbox{}}


$if(verbatim-in-note)$
    \usepackage{fancyvrb}
    \VerbatimFootnotes
$endif$

$if(verbatim-in-note)$
    \VerbatimFootnotes % allows verbatim text in footnotes
$endif$

% use for subfigures
% set up image load path
% Figures
\usepackage{graphicx,grffile}
\graphicspath{ {../img/} }

\usepackage{caption}
\usepackage{subcaption}

\usepackage{float}
\floatstyle{boxed}
\restylefloat{figure}


\makeatletter
\def\maxwidth{\ifdim\Gin@nat@width>\linewidth\linewidth\else\Gin@nat@width\fi}
\def\maxheight{\ifdim\Gin@nat@height>\textheight\textheight\else\Gin@nat@height\fi}
\makeatother
% Scale images if necessary, so that they will not overflow the page
% margins by default, and it is still possible to overwrite the defaults
% using explicit options in \includegraphics[width, height, ...]{}
\setkeys{Gin}{width=\maxwidth,height=\maxheight,keepaspectratio}
$if(tables)$
    \usepackage{longtable,booktabs}
$endif$
\usepackage{multirow}
% set header and footer
% use default style, if don't need this, just comment the package
\usepackage{fancyhdr}
\pagestyle{plain}
%  left (L), right (R) and centre (C), odd (O) or even (E) pa
% \fancyhead{}
% clear the header
% \fancyhead[RO,LE]{Thesis Title}
% \fancyfoot{}
% \fancyfoot[LE,RO]{\thepage}
% \fancyfoot[LO,CE]{Chapter \thechapter}
% \fancyfoot[CO,RE]{Author Name}


% change the thickness of the lines in the headers and footers we use this code entering a size in points:
\renewcommand{\headrulewidth}{0.4pt}
\renewcommand{\footrulewidth}{0.4pt}
$if(biblatex)$
    \usepackage[backend=bibtex, style=numeric, sorting=nty]{biblatex}
    $if(biblio-files)$
        %\bibliography{$biblio-files$}
        \addbibresource{$biblio-files$}
    $endif$
$endif$
% wallpaper
\usepackage{wallpaper}
\makeatletter

\newcommand\frontmatter{
    \cleardoublepage
    \pagenumbering{roman}
}

\newcommand\mainmatter{
    \cleardoublepage
    \pagenumbering{arabic}
}

\newcommand\backmatter{
    \if@openright
        \cleardoublepage
    \else
        \clearpage
    \fi
}
% Add clickable links to a section/subsection with a PDF document
\usepackage{hyperref}
\hypersetup{
    colorlinks   = true,    % Colours links instead of ugly boxes
    urlcolor     = blue,    % Colour for external hyperlinks
    linkcolor    = blue,    % Colour of internal links
    citecolor    = red      % Colour of citations
}

\urlstyle{same}

$if(links-as-notes)$
    % Make links footnotes instead of hotlinks:
    \renewcommand{\href}[2]{#2\footnote{\url{#1}}}
$endif$
$for(header-includes)$
    $header-includes$
$endfor$

% use upquote if available, for straight quotes in verbatim environments
\IfFileExists{upquote.sty}{\usepackage{upquote}}{}
% use microtype if available
\IfFileExists{microtype.sty}{%
    \usepackage{microtype}
    \UseMicrotypeSet[protrusion]{basicmath} % disable protrusion for tt fonts
}{}
\date{\today}

% 月 轉換成對應 英文月 單字 1 -> January
\newcommand{\MONTH}{
    \ifcase\the\month
        \or January% 1
        \or February% 2
        \or March% 3
        \or April% 4
        \or May% 5
        \or June% 6
        \or July% 7
        \or August% 8
        \or September% 9
        \or October% 10
        \or November% 11
        \or December% 12
    \fi
}

% 計算民國幾年
\usepackage{calc}
\newcounter{tyear}
\setcounter{tyear}{\year-1911}

% 民國 \yearzh{} 年 \monthzh{} 月
\newcommand{\yearzh}{\thetyear}
\newcommand{\monthzh}{\the\month}

% 西元 \monthen{}, \monthen{}
\newcommand{\yearen}{\the\year}
\newcommand{\monthen}{\MONTH}

\begin{document}


    \begin{titlepage}
        \vspace{1cm}
        \begin{singlespace}
        \begin{center}
            \fontsize{36}{54}\selectfont{
                $university_zh$\par
            }
            \fontsize{28}{42}\selectfont{
                $institute_zh$\par
            }

            \fontsize{24}{36}\selectfont{
                碩士論文\par
            }

            \fontsize{20}{30}\selectfont{
                $title_zh$\par
                $title_en$\par
            }

            \vspace{\fill}

            \fontsize{18}{27}\selectfont{
                研究生:$author_zh$\par
                指導教授:$advisor_zh$\par
            }

            \vspace{1.5cm}
            \fontsize{16}{24}\selectfont{
                中華民國 \thetyear{} 年 \monthzh{} 月\par
                %\monthen{}, \yearen{} \par
            }
        \end{center}
        \end{singlespace}
        \vspace{1cm}
    \end{titlepage}

    \begin{titlepage}
        \vspace{1cm}
        \begin{singlespace}
            \begin{center}
                \fontsize{20}{30}\selectfont{
                    $title_zh$\par
                    $title_en$\par
                }
                \fontsize{14}{21}\selectfont{
                    研究生:$author_zh$ \hfill Student : $author_en$\\
                    指導教授:$advisor_zh$ \hfill Advisor : $advisor_en$\par
                }
                \vspace{1.5cm}
                \fontsize{14}{21}\selectfont{
                    $university_zh$\par
                    $institute_zh$\par
                    碩士論文\par
                    A  Thesis Submitted to $institute_en$\par
                    $collage_en$\par
                    $university_en$\par
                    in Partial Fulfillment of the Requirements\par
                    for the Degree of Master of Science in\par
                    $institute_en$\par
                }
                \vspace{\fill}
                \vspace{1.5cm}
                \fontsize{12}{20}\selectfont{
                    \monthen{}, \yearen{}\par
                    Huwei, Yunlin, Taiwan, Republic of China\par
                }
                \fontsize{16}{24}\selectfont{
                    中華民國 \thetyear{} 年 \monthzh{} 月\par
                }
            \end{center}
        \end{singlespace}
        \vspace{1cm}
    \end{titlepage}


    % 插入浮水印
    \CenterWallPaper{0.5}{watermark.jpg}

    % set front page
    % and set page number to arabic i ii...
    % \pagenumbering{arabic}
    \frontmatter

    \chapter*{$title_zh$}
    \addcontentsline{toc}{chapter}{中文摘要}
    \begin{center}
        \vspace{0.4cm}
        \fontsize{14}{21}\selectfont
        學生:$author_zh$ \hfill 指導教授:$advisor_zh$\par
        \vspace{1cm}
        \fontsize{14}{21}\selectfont{
            $university_zh$ $institute_zh$\par
        }
        \vspace{0.9cm}
        \fontsize{20}{30}\selectfont{
            \textbf{摘要}
        }
    \end{center}
    % the abstract content
    $abstract_zh$


    \chapter*{$title_en$}
    \addcontentsline{toc}{chapter}{英文摘要}
    \begin{center}
        \vspace{0.4cm}
        \fontsize{14}{21}\selectfont
        Student:$author_en$ \hfill Advisor:$advisor_en$\\
        \vspace{1cm}
        \fontsize{14}{21}\selectfont{
            $institute_en$\\$university_en$\par
        }
        \vspace{0.9cm}
        \fontsize{20}{30}\selectfont{
            \textbf{Abstract}
        }
    \end{center}
    % the abstract content
    $abstract_en$

    $for(include-before)$
        $include-before$
    $endfor$

    % table of content
    $if(toc)$
        \begingroup
            \singlespacing
            % table of content
            \tableofcontents

            % figure table
            \renewcommand{\numberline}[1]{\loflabel~#1\hspace*{1em}}
            \listoffigures

            % table of table
            \renewcommand{\numberline}[1]{\lotlabel~#1\hspace*{1em}}
            \listoftables
        \endgroup
    $endif$
    
    % start normal page number, 1 2 3
    \mainmatter
        
    $body$
    
    \cleardoublepage
    
    $if(biblatex)$
        \renewcommand\bibname{參考文獻}
        \addcontentsline{toc}{chapter}{參考文獻}
        \printbibliography
    $endif$
    
    \newpage
    
    \renewcommand\appendixtocname{附錄}
    \renewcommand\appendixpagename{附錄}

    \appendix
    \appendixpage
    $for(include-after)$
    $include-after$

    $endfor$

\end{document}
